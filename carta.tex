Prezado Prof. Marcelo Colaço,

\bigskip
Durante a Graduação na Faculdade de Engenharia da UERJ (FEM/UERJ),
tive a oportunidade de aprender sobre os princípios de conservação
que governam os Fenômenos de Transporte.
Em seguida, tive a oportunidade de iniciar uma pesquisa científica
orientada pelo Prof. Gustavo Rabello dos Anjos e 
pelo Prof. José da Rocha Miranda Pontes,
na qual tínhamos por objetivo realizar simulações numéricas
em uma artéria coronária utilizando o Método de Elementos Finitos.
Essa pesquisa científica despertou o meu grande interesse em
trabalhar com pesquisa científica e trouxe uma grande oportunidade
de aprendizado em Elementos Finitos e em Linguagem de Programação.

\medskip
Após a conclusão da Graduação, iniciei o curso de Mestrado no 
Programa de Pós-Graduação de Engenharia Mecânica da UERJ (PPGEM-UERJ),
onde tive a oportunidade em me aprofundar
na Mecânica dos Fluidos Computacional e na Análise Computacional de
Sistemas Lineares.
Durante o curso de Mestrado, continuamos a pesquisa em Sistemas Biológicos,
porém foram implementados importantes métodos que aprimoraram a capacidade
do código computacional, tais como: 
\textit{Arbitrary Lagrangian-Eulerian} (ALE),
\textit{semi-Lagrangian} e
\textit{Gaussian Quadrature}.

\medskip
Sendo assim, durante o curso de Doutorado no Programa de Engenharia Mecânica da COPPE/UFRJ, 
devido a sua excelência acadêmica e 
a ampla infraestrutura de recursos computacionais,
espero desenvolver o conhecimento em Escoamentos Multifásicos e Fluidos não-Newtonianos. 
O curso de Doutorado na PEM/COPPE, dessa forma, tende a proporcionar uma grande oportunidade 
de aprofundar a pesquisa em Sistemas Biológicos de elevada complexidade
e, com isso, a geração de publicações em canais de comunicação nacionais
e internacionais de excelência e alto impacto.

\medskip
Portanto, venho por meio desta carta solicitar a minha inscrição
no curso de Doutorado do Programa de Engenharia Mecânica
da COPPE (PEM/COPPE) para o terceiro período de 2020,
na área de concentração de Termociências e Engenharia
Térmica.




\vspace{1cm}
\hspace{7.5cm}\includegraphics[scale=0.14]{figure/assinatura.png}

\begin{center}
\vspace{-1.2cm}
\line(1,0){260}

\smallskip
Leandro Marques dos Santos, M.Sc.

\medskip
\small 2 de Julho de 2020\\
\small Rio de Janeiro, Brasil

\end{center}

